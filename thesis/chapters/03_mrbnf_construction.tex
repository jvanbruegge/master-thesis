\chapter{MRBNF Construction}

\section{Parsing of user specification}

\begin{itemize}
\item{Not clear yet what would be best syntactic format}
\item{Example on how such a format might look, maybe mention Nominal2?}
\item{Rather easy after automation of proofs}
\item{Thus skipped for this thesis}
\end{itemize}

\section{Definition of a composite type}

\begin{itemize}
\item{Uses Isabelle's \texttt{typedef}}
\item{Variables instead of recursion}
\item{Final version: Generated from the user spec}
\item{This thesis: Defined by the user}
\end{itemize}

\section{Definition of the pre-datatype by composition}

\begin{itemize}
\item{Partly implemented by roshardt~\cite{mrbnf_composition}}
\end{itemize}

\subsection{Construction of a MRBNF from a type}

\begin{itemize}
\item{Decend into composite type using recursion}
\item{if t = type var, then return ID, possibly demoted, see section~\ref{sec:demote}}
\item{if t = (t1...tn) F is a BNF, convert to MRBNF (see section~\ref{sec:bnf_to_mrbnf}) and go to next step}
\item{if t = (t1...tn) F is a MRBNF, create MRBNFs for t1...tn by recursion, rearrange vars and compose with F}
\item{if t = (t1...tn) F else, return DEADID}
\end{itemize}

\subsection{Conversion from BNF to MRBNF}\label{sec:bnf_to_mrbnf}

\begin{itemize}
\item{MRBNF is generalization of BNF --> every BNF is a MRBNF}
\item{Only live and dead variables, no free or bound}
\end{itemize}

\subsection{Demotion of variables}\label{sec:demote}

\begin{itemize}
\item{Converts live --> free --> bound --> dead}
\item{Needed to match vars of multiple MRBNFs during composition}
\end{itemize}

\subsection{Lifing and Permutation of variables}\label{sec:lift}

\begin{itemize}
\item{Lift: Add new (unused) variables to the front of the type}
\item{Used to make multiple MRBNFs have the same number of variables of each type}
\item{Permute: Change order of variables of a MRBNF}
\item{Composition needs all variables in the same order for each MRBNF}
\end{itemize}

\subsection{MRBNF composition}

\begin{itemize}
\item{Explain how to combine axioms of multiple MRBNFs to one composed MRBNF}
\end{itemize}

\section{Fixpoint construction of the data type}

\begin{itemize}
\item{Partly implemented by Stoop~\cite{mrbnf_fixpoint}}
\item{Use the greatest or least fixpoint of the pre-datatype to obtain a recursive datatype}
\item{Generates a recursor/corecursor that can be used to write functions}
\item{Generates an weak induction principle}
\end{itemize}
