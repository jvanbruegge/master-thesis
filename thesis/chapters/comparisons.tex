\chapter{Comparison with existing solutions}\label{chapter:comparison}

Over the years, three main categories of variable representations have emerged: Nameless and syntactic representations as well as representations based on nominal logic\cite{nominal_logic} (nameful). Some reasoning frameworks also combine multiple paradigms like locally nameless\cite{locally_nameless}. \acp{MRBNF} firmly lie in the nominal logic category. However, the idea of representing variable binders as shapes to be filled is inspired by \ac{HOAS}.

\section{Nameless Representations}

De Bruijn indices\cite{deBruijn} and their variants circumvent the issue of variables names completely. Instead of using names to refer to free variables, in a nameless representation a free variable is a natural number that specifies how many enclosing lambda abstractions should be skipped. For example the $I$ combinator $\lambda x. \: x$ would be represented as $\lambda. \: 0$ while the $K$ combinator $\lambda x. \: \lambda y. \: x$ would be represented as $\lambda. \: \lambda. 1$. This has the advantage that $\alpha$-equivalence is the same as syntactic equivalence, making it easy to implement.

However handling free variables and shifting variable references during substitution is harder than the nameful approach. Also it is not easy to generalize to more complex binding patterns like linear binders\footnote{binding the same variable multiple times in the same binder}, unordered simultaneous binders or recursive binders.

\section{Nominal Logic}

At the time of writing, the most advanced implementation of nominal logic is the Nominal2 package~\cite{nominal2} for Isabelle. It supports simple and recursive binders and provides a custom induction scheme with freshness assumptions built in. To work with deep binders (like the record patterns in task 1B and 2B of the POPLmark challenge) it allows to define a function to extract variables from a datatype (see listing~\ref{lst:letrec_nominal}). Despite the support for more complex binding patterns, Nominal2 still has major limitations. For example it is not possible to express non-repetitive binders and it is not possible to nest recursion through normal Isabelle data types like \texttt{list}. This forces the user to create auxiliary, mutually recursive datatypes and thus they loose the rich library of theorems that already exists for the standard \texttt{list} type. Additionally, defining a function on a nominal datatype requires a lot of user written proofs to uphold variable freshness. The recursor for \acp{MRBNF} will have these proofs already built in.

\begin{lstlisting}[
  language=Isabelle,
  caption=Defintion of a simple expression type with recursive lets in Nominal2,
  label={lst:letrec_nominal}
]
atom_decl var

nominal_datatype expr =
    Var var
  | Lam "x::var" "t::expr" binds x in t
  | App "t1::expr" "t2::expr"
  | Let "as::binder_list" "t::expr" binds bn(as) in t
  | LetRec "as::binder_list" "t::expr" binds bn(as) in as t
and binder_list =
    ANil
  | ACons var expr binder_list
binder bn :: "binder_list => atom list" where
    "bn ANil = []"
  | "bn (ACons x _ as) = atom x # bn as"
\end{lstlisting}

\section{Higher-order abstract syntax}

The main idea of \ac{HOAS} is to use the binding mechanism of the meta language to express the binding shape of the object language. Variables are represented by functions that can be seen as continuations that fill a shape with data. It is popular as standalone framework in languages like Twelf\cite{twelf}, but also implemented in general purpose theorem provers like Coq or Isabelle.

